%% Generated by Sphinx.
\def\sphinxdocclass{report}
\documentclass[letterpaper,10pt,english]{sphinxmanual}
\ifdefined\pdfpxdimen
   \let\sphinxpxdimen\pdfpxdimen\else\newdimen\sphinxpxdimen
\fi \sphinxpxdimen=.75bp\relax
\ifdefined\pdfimageresolution
    \pdfimageresolution= \numexpr \dimexpr1in\relax/\sphinxpxdimen\relax
\fi
%% let collapsible pdf bookmarks panel have high depth per default
\PassOptionsToPackage{bookmarksdepth=5}{hyperref}

\PassOptionsToPackage{warn}{textcomp}
\usepackage[utf8]{inputenc}
\ifdefined\DeclareUnicodeCharacter
% support both utf8 and utf8x syntaxes
  \ifdefined\DeclareUnicodeCharacterAsOptional
    \def\sphinxDUC#1{\DeclareUnicodeCharacter{"#1}}
  \else
    \let\sphinxDUC\DeclareUnicodeCharacter
  \fi
  \sphinxDUC{00A0}{\nobreakspace}
  \sphinxDUC{2500}{\sphinxunichar{2500}}
  \sphinxDUC{2502}{\sphinxunichar{2502}}
  \sphinxDUC{2514}{\sphinxunichar{2514}}
  \sphinxDUC{251C}{\sphinxunichar{251C}}
  \sphinxDUC{2572}{\textbackslash}
\fi
\usepackage{cmap}
\usepackage[T1]{fontenc}
\usepackage{amsmath,amssymb,amstext}
\usepackage{babel}



\usepackage{tgtermes}
\usepackage{tgheros}
\renewcommand{\ttdefault}{txtt}



\usepackage[Bjarne]{fncychap}
\usepackage{sphinx}

\fvset{fontsize=auto}
\usepackage{geometry}


% Include hyperref last.
\usepackage{hyperref}
% Fix anchor placement for figures with captions.
\usepackage{hypcap}% it must be loaded after hyperref.
% Set up styles of URL: it should be placed after hyperref.
\urlstyle{same}


\usepackage{sphinxmessages}
\setcounter{tocdepth}{1}



\title{prpy: Probabilistic Robot Localization Python Library}
\date{Nov 14, 2023}
\release{0.1}
\author{Pere Ridao}
\newcommand{\sphinxlogo}{\vbox{}}
\renewcommand{\releasename}{Release}
\makeindex
\begin{document}

\pagestyle{empty}
\sphinxmaketitle
\pagestyle{plain}
\sphinxtableofcontents
\pagestyle{normal}
\phantomsection\label{\detokenize{index::doc}}


\sphinxAtStartPar
\sphinxstylestrong{Probabilistic Robot Localization} is Python Library containing the main algorithms explained in the \sphinxstylestrong{Probabilisitic Robot Localization} Book used in the \sphinxstylestrong{Probabilisitic Robotics} and the \sphinxstylestrong{Hands\sphinxhyphen{}on Localization} Courses of the \sphinxstylestrong{Intelligent Field Robotic Systems (IFRoS)} European Erasmus Mundus Master.

\begin{sphinxadmonition}{note}{Note:}
\sphinxAtStartPar
This documentation is still under construction.
\end{sphinxadmonition}


\chapter{API:}
\label{\detokenize{index:api}}

\section{Pose Representation}
\label{\detokenize{compounding:pose-representation}}\label{\detokenize{compounding::doc}}

\subsection{Pose 3DOF}
\label{\detokenize{compounding:pose-3dof}}
\begin{figure}[htbp]
\centering

\noindent\sphinxincludegraphics[scale=0.75]{{Pose3D}.png}
\end{figure}
\index{Pose3D (class in Pose3D)@\spxentry{Pose3D}\spxextra{class in Pose3D}}

\begin{fulllineitems}
\phantomsection\label{\detokenize{compounding:Pose3D.Pose3D}}\pysiglinewithargsret{\sphinxbfcode{\sphinxupquote{class\DUrole{w}{  }}}\sphinxcode{\sphinxupquote{Pose3D.}}\sphinxbfcode{\sphinxupquote{Pose3D}}}{\emph{\DUrole{n}{input\_array}}}{}
\sphinxAtStartPar
Bases: \sphinxcode{\sphinxupquote{numpy.ndarray}}

\sphinxAtStartPar
Definition of a robot pose in 3 DOF (x, y, yaw). The class inherits from a ndarray.
This class extends the ndarray with the \$oplus\$ and \$ominus\$ operators and the corresponding Jacobians.
\index{oplus() (Pose3D.Pose3D method)@\spxentry{oplus()}\spxextra{Pose3D.Pose3D method}}

\begin{fulllineitems}
\phantomsection\label{\detokenize{compounding:Pose3D.Pose3D.oplus}}\pysiglinewithargsret{\sphinxbfcode{\sphinxupquote{oplus}}}{\emph{\DUrole{n}{BxC}}}{}
\sphinxAtStartPar
Given a Pose3D object \sphinxstyleemphasis{AxB} (the self object) and a Pose3D object \sphinxstyleemphasis{BxC}, it returns the Pose3D object \sphinxstyleemphasis{AxC}.
\begin{equation*}
\begin{split}\mathbf{{^A}x_B} &= \begin{bmatrix} ^Ax_B & ^Ay_B & ^A\psi_B \end{bmatrix}^T \\
\mathbf{{^B}x_C} &= \begin{bmatrix} ^Bx_C & ^By_C & & ^B\psi_C \end{bmatrix}^T \\\end{split}
\end{equation*}
\sphinxAtStartPar
The operation is defined as:
\begin{equation}\label{equation:compounding:eq-oplus3dof}
\begin{split}\mathbf{{^A}x_C} &= \mathbf{{^A}x_B} \oplus \mathbf{{^B}x_C} =
\begin{bmatrix}
    ^Ax_B + ^Bx_C  \cos(^A\psi_B) - ^By_C  \sin(^A\psi_B) \\
    ^Ay_B + ^Bx_C  \sin(^A\psi_B) + ^By_C  \cos(^A\psi_B) \\
    ^A\psi_B + ^B\psi_C
\end{bmatrix}\end{split}
\end{equation}\begin{quote}\begin{description}
\item[{Parameters}] \leavevmode
\sphinxAtStartPar
\sphinxstyleliteralstrong{\sphinxupquote{BxC}} \textendash{} C\sphinxhyphen{}Frame pose expressed in B\sphinxhyphen{}Frame coordinates

\item[{Returns}] \leavevmode
\sphinxAtStartPar
C\sphinxhyphen{}Frame pose expressed in A\sphinxhyphen{}Frame coordinates

\end{description}\end{quote}

\end{fulllineitems}

\index{ominus() (Pose3D.Pose3D method)@\spxentry{ominus()}\spxextra{Pose3D.Pose3D method}}

\begin{fulllineitems}
\phantomsection\label{\detokenize{compounding:Pose3D.Pose3D.ominus}}\pysiglinewithargsret{\sphinxbfcode{\sphinxupquote{ominus}}}{}{}
\sphinxAtStartPar
Inverse pose compounding of the \sphinxstyleemphasis{AxB} pose (the self objetc):
\begin{equation}\label{equation:compounding:eq-ominus3dof}
\begin{split}^Bx_A = \ominus ^Ax_B =
\begin{bmatrix}
    -^Ax_B \cos(^A\psi_B) - ^Ay_B \sin(^A\psi_B) \\
    ^Ax_B \sin(^A\psi_B) - ^Ay_B \cos(^A\psi_B) \\
    -^A\psi_B
\end{bmatrix}\end{split}
\end{equation}\begin{quote}\begin{description}
\item[{Returns}] \leavevmode
\sphinxAtStartPar
A\sphinxhyphen{}Frame pose expressed in B\sphinxhyphen{}Frame coordinates (eq. \eqref{equation:compounding:eq-ominus3dof})

\end{description}\end{quote}

\end{fulllineitems}


\end{fulllineitems}



\section{Robot Simulation}
\label{\detokenize{robot_simulation:robot-simulation}}\label{\detokenize{robot_simulation::doc}}
\begin{figure}[htbp]
\centering
\capstart

\noindent\sphinxincludegraphics[scale=0.75]{{SimulatedRobot}.png}
\caption{SimulatedRobot Class Diagram.}\label{\detokenize{robot_simulation:id1}}\end{figure}
\index{SimulatedRobot (class in SimulatedRobot)@\spxentry{SimulatedRobot}\spxextra{class in SimulatedRobot}}

\begin{fulllineitems}
\phantomsection\label{\detokenize{robot_simulation:SimulatedRobot.SimulatedRobot}}\pysiglinewithargsret{\sphinxbfcode{\sphinxupquote{class\DUrole{w}{  }}}\sphinxcode{\sphinxupquote{SimulatedRobot.}}\sphinxbfcode{\sphinxupquote{SimulatedRobot}}}{\emph{\DUrole{n}{xs0}}, \emph{\DUrole{n}{map}\DUrole{o}{=}\DUrole{default_value}{{[}{]}}}, \emph{\DUrole{o}{*}\DUrole{n}{args}}}{}
\sphinxAtStartPar
Bases: \sphinxcode{\sphinxupquote{object}}

\sphinxAtStartPar
This is the base class to simulate a robot. There are two operative frames: the world  N\sphinxhyphen{}Frame (North East Down oriented) and the robot body frame body B\sphinxhyphen{}Frame.
Each robot has a motion model and a measurement model. The motion model is used to simulate the robot motion and the measurement model is used to simulate the robot measurements.

\sphinxAtStartPar
\sphinxstylestrong{All Robot simulation classes must derive from this class} .
\index{dt (SimulatedRobot.SimulatedRobot attribute)@\spxentry{dt}\spxextra{SimulatedRobot.SimulatedRobot attribute}}

\begin{fulllineitems}
\phantomsection\label{\detokenize{robot_simulation:SimulatedRobot.SimulatedRobot.dt}}\pysigline{\sphinxbfcode{\sphinxupquote{dt}}\sphinxbfcode{\sphinxupquote{\DUrole{w}{  }\DUrole{p}{=}\DUrole{w}{  }0.1}}}
\sphinxAtStartPar
class attribute containing sample time of the simulation

\end{fulllineitems}

\index{\_\_init\_\_() (SimulatedRobot.SimulatedRobot method)@\spxentry{\_\_init\_\_()}\spxextra{SimulatedRobot.SimulatedRobot method}}

\begin{fulllineitems}
\phantomsection\label{\detokenize{robot_simulation:SimulatedRobot.SimulatedRobot.__init__}}\pysiglinewithargsret{\sphinxbfcode{\sphinxupquote{\_\_init\_\_}}}{\emph{\DUrole{n}{xs0}}, \emph{\DUrole{n}{map}\DUrole{o}{=}\DUrole{default_value}{{[}{]}}}, \emph{\DUrole{o}{*}\DUrole{n}{args}}}{}\begin{quote}\begin{description}
\item[{Parameters}] \leavevmode\begin{itemize}
\item {} 
\sphinxAtStartPar
\sphinxstyleliteralstrong{\sphinxupquote{xs0}} \textendash{} initial simulated robot state \(x_{s_0}\) used to initialize the the motion model

\item {} 
\sphinxAtStartPar
\sphinxstyleliteralstrong{\sphinxupquote{map}} \textendash{} feature map of the environment \(M=[^Nx_{F_1}^T,...,^Nx_{F_{nf}}^T]^T\)

\end{itemize}

\end{description}\end{quote}

\sphinxAtStartPar
Constructor. First, it initializes the robot simulation defining the following attributes:
\begin{itemize}
\item {} 
\sphinxAtStartPar
\sphinxstylestrong{k} : time step

\item {} 
\sphinxAtStartPar
\sphinxstylestrong{Qsk} : \sphinxstylestrong{To be defined in the derived classes}. Object attribute containing Covariance of the simulation motion model noise

\item {} 
\sphinxAtStartPar
\sphinxstylestrong{usk} : \sphinxstylestrong{To be defined in the derived classes}. Object attribute contining the simulated input to the motion model

\item {} 
\sphinxAtStartPar
\sphinxstylestrong{xsk} : \sphinxstylestrong{To be defined in the derived classes}. Object attribute contining the current simulated robot state

\item {} 
\sphinxAtStartPar
\sphinxstylestrong{zsk} : \sphinxstylestrong{To be defined in the derived classes}. Object attribute contining the current simulated robot measurement

\item {} 
\sphinxAtStartPar
\sphinxstylestrong{Rsk} : \sphinxstylestrong{To be defined in the derived classes}. Object attribute contining the observation noise covariance matrix

\item {} 
\sphinxAtStartPar
\sphinxstylestrong{xsk} : current pose is the initial state

\item {} 
\sphinxAtStartPar
\sphinxstylestrong{xsk\_1} : previouse state is the initial robot state

\item {} 
\sphinxAtStartPar
\sphinxstylestrong{M} : position of the features in the N\sphinxhyphen{}Frame

\item {} 
\sphinxAtStartPar
\sphinxstylestrong{nf} : number of features

\end{itemize}

\sphinxAtStartPar
Then, the robot animation is initialized defining the following attributes:
\begin{itemize}
\item {} 
\sphinxAtStartPar
\sphinxstylestrong{vehicleIcon} : Path file of the image of the robot to be used in the animation

\item {} 
\sphinxAtStartPar
\sphinxstylestrong{vehicleFig} : Figure of the robot to be used in the animation

\item {} 
\sphinxAtStartPar
\sphinxstylestrong{vehicleAxes} : Axes of the robot to be used in the animation

\item {} 
\sphinxAtStartPar
\sphinxstylestrong{xTraj} : list containing the x coordinates of the robot trajectory

\item {} 
\sphinxAtStartPar
\sphinxstylestrong{yTraj} : list containing the y coordinates of the robot trajectory

\item {} 
\sphinxAtStartPar
\sphinxstylestrong{visualizationInterval} : time\sphinxhyphen{}steps interval between two consecutive frames of the animation

\end{itemize}

\end{fulllineitems}

\index{PlotRobot() (SimulatedRobot.SimulatedRobot method)@\spxentry{PlotRobot()}\spxextra{SimulatedRobot.SimulatedRobot method}}

\begin{fulllineitems}
\phantomsection\label{\detokenize{robot_simulation:SimulatedRobot.SimulatedRobot.PlotRobot}}\pysiglinewithargsret{\sphinxbfcode{\sphinxupquote{PlotRobot}}}{}{}
\sphinxAtStartPar
Updates the plot of the robot at the current pose

\end{fulllineitems}

\index{fs() (SimulatedRobot.SimulatedRobot method)@\spxentry{fs()}\spxextra{SimulatedRobot.SimulatedRobot method}}

\begin{fulllineitems}
\phantomsection\label{\detokenize{robot_simulation:SimulatedRobot.SimulatedRobot.fs}}\pysiglinewithargsret{\sphinxbfcode{\sphinxupquote{fs}}}{\emph{\DUrole{n}{xsk\_1}}, \emph{\DUrole{n}{uk}}}{}
\sphinxAtStartPar
Motion model used to simulate the robot motion. Computes the current robot state \(x_k\) given the previous robot state \(x_{k-1}\) and the input \(u_k\).
It also updates the object attributes \(xsk\), \(xsk_1\) and  \(usk\) to be made them available for plotting purposes.
\sphinxstyleemphasis{To be overriden in child class}.
\begin{quote}\begin{description}
\item[{Parameters}] \leavevmode\begin{itemize}
\item {} 
\sphinxAtStartPar
\sphinxstyleliteralstrong{\sphinxupquote{xsk\_1}} \textendash{} previous robot state \(x_{k-1}\)

\item {} 
\sphinxAtStartPar
\sphinxstyleliteralstrong{\sphinxupquote{usk}} \textendash{} model input \(u_{s_k}\)

\end{itemize}

\item[{Returns}] \leavevmode
\sphinxAtStartPar
current robot state \(x_k\)

\end{description}\end{quote}

\end{fulllineitems}

\index{SetMap() (SimulatedRobot.SimulatedRobot method)@\spxentry{SetMap()}\spxextra{SimulatedRobot.SimulatedRobot method}}

\begin{fulllineitems}
\phantomsection\label{\detokenize{robot_simulation:SimulatedRobot.SimulatedRobot.SetMap}}\pysiglinewithargsret{\sphinxbfcode{\sphinxupquote{SetMap}}}{\emph{\DUrole{n}{map}}}{}
\sphinxAtStartPar
Initializes the map of the environment.

\end{fulllineitems}

\index{\_PlotSample() (SimulatedRobot.SimulatedRobot method)@\spxentry{\_PlotSample()}\spxextra{SimulatedRobot.SimulatedRobot method}}

\begin{fulllineitems}
\phantomsection\label{\detokenize{robot_simulation:SimulatedRobot.SimulatedRobot._PlotSample}}\pysiglinewithargsret{\sphinxbfcode{\sphinxupquote{\_PlotSample}}}{\emph{\DUrole{n}{x}}, \emph{\DUrole{n}{P}}, \emph{\DUrole{n}{n}}}{}
\sphinxAtStartPar
Plots n samples of a multivariate gaussian distribution. This function is used only for testing, to plot the
uncertainty through samples.
:param x: mean pose of the distribution
:param P: covariance of the distribution
:param n: number of samples to plot

\end{fulllineitems}


\end{fulllineitems}



\subsection{3 DOF Diferential Drive Robot Simulation}
\label{\detokenize{robot_simulation:dof-diferential-drive-robot-simulation}}
\begin{figure}[htbp]
\centering
\capstart

\noindent\sphinxincludegraphics[scale=0.75]{{DifferentialDriveSimulatedRobot}.png}
\caption{DifferentialDriveSimulatedRobot Class Diagram.}\label{\detokenize{robot_simulation:id2}}\end{figure}
\index{DifferentialDriveSimulatedRobot (class in DifferentialDriveSimulatedRobot)@\spxentry{DifferentialDriveSimulatedRobot}\spxextra{class in DifferentialDriveSimulatedRobot}}

\begin{fulllineitems}
\phantomsection\label{\detokenize{robot_simulation:DifferentialDriveSimulatedRobot.DifferentialDriveSimulatedRobot}}\pysiglinewithargsret{\sphinxbfcode{\sphinxupquote{class\DUrole{w}{  }}}\sphinxcode{\sphinxupquote{DifferentialDriveSimulatedRobot.}}\sphinxbfcode{\sphinxupquote{DifferentialDriveSimulatedRobot}}}{\emph{\DUrole{n}{xs0}}, \emph{\DUrole{n}{map}\DUrole{o}{=}\DUrole{default_value}{{[}{]}}}, \emph{\DUrole{o}{*}\DUrole{n}{args}}}{}
\sphinxAtStartPar
Bases: {\hyperref[\detokenize{robot_simulation:SimulatedRobot.SimulatedRobot}]{\sphinxcrossref{\sphinxcode{\sphinxupquote{SimulatedRobot.SimulatedRobot}}}}}

\sphinxAtStartPar
This class implements a simulated differential drive robot. It inherits from the \sphinxcode{\sphinxupquote{SimulatedRobot}} class and
overrides some of its methods to define the differential drive robot motion model.
\index{\_\_init\_\_() (DifferentialDriveSimulatedRobot.DifferentialDriveSimulatedRobot method)@\spxentry{\_\_init\_\_()}\spxextra{DifferentialDriveSimulatedRobot.DifferentialDriveSimulatedRobot method}}

\begin{fulllineitems}
\phantomsection\label{\detokenize{robot_simulation:DifferentialDriveSimulatedRobot.DifferentialDriveSimulatedRobot.__init__}}\pysiglinewithargsret{\sphinxbfcode{\sphinxupquote{\_\_init\_\_}}}{\emph{\DUrole{n}{xs0}}, \emph{\DUrole{n}{map}\DUrole{o}{=}\DUrole{default_value}{{[}{]}}}, \emph{\DUrole{o}{*}\DUrole{n}{args}}}{}\begin{quote}\begin{description}
\item[{Parameters}] \leavevmode\begin{itemize}
\item {} 
\sphinxAtStartPar
\sphinxstyleliteralstrong{\sphinxupquote{xs0}} \textendash{} initial simulated robot state \(\mathbf{x_{s_0}}=[^Nx{_{s_0}}~^Ny{_{s_0}}~^N\psi{_{s_0}}~]^T\) used to initialize the  motion model

\item {} 
\sphinxAtStartPar
\sphinxstyleliteralstrong{\sphinxupquote{map}} \textendash{} feature map of the environment \(M=[^Nx_{F_1},...,^Nx_{F_{nf}}]\)

\end{itemize}

\end{description}\end{quote}

\sphinxAtStartPar
Initializes the simulated differential drive robot. Overrides some of the object attributes of the parent class \sphinxcode{\sphinxupquote{SimulatedRobot}} to define the differential drive robot motion model:
\begin{itemize}
\item {} 
\sphinxAtStartPar
\sphinxstylestrong{Qsk} : Object attribute containing Covariance of the simulation motion model noise.

\end{itemize}
\begin{equation}\label{equation:robot_simulation:eq:Qsk}
\begin{split}Q_k=\begin{bmatrix}\sigma_{\dot u}^2 & 0 & 0\\
0 & \sigma_{\dot v}^2 & 0 \\
0 & 0 & \sigma_{\dot r}^2 \\
\end{bmatrix}\end{split}
\end{equation}\begin{itemize}
\item {} 
\sphinxAtStartPar
\sphinxstylestrong{usk} : Object attribute containing the simulated input to the motion model containing the forward velocity \(u_k\) and the angular velocity \(r_k\)

\end{itemize}
\begin{equation}\label{equation:robot_simulation:eq:usk}
\begin{split}\bf{u_k}=\begin{bmatrix}u_k & r_k\end{bmatrix}^T\end{split}
\end{equation}\begin{itemize}
\item {} 
\sphinxAtStartPar
\sphinxstylestrong{xsk} : Object attribute containing the current simulated robot state

\end{itemize}
\begin{equation}\label{equation:robot_simulation:eq:xsk}
\begin{split}x_k=\begin{bmatrix}{^N}x_k & {^N}y_k & {^N}\theta_k & {^B}u_k & {^B}v_k & {^B}r_k\end{bmatrix}^T\end{split}
\end{equation}
\sphinxAtStartPar
where \({^N}x_k\), \({^N}y_k\) and \({^N}\theta_k\) are the robot position and orientation in the world N\sphinxhyphen{}Frame, and \({^B}u_k\), \({^B}v_k\) and \({^B}r_k\) are the robot linear and angular velocities in the robot B\sphinxhyphen{}Frame.
\begin{itemize}
\item {} 
\sphinxAtStartPar
\sphinxstylestrong{zsk} : Object attribute containing \(z_{s_k}=[n_L~n_R]^T\) observation vector containing number of pulses read from the left and right wheel encoders.

\item {} 
\sphinxAtStartPar
\sphinxstylestrong{Rsk} : Object attribute containing \(R_{s_k}=diag(\sigma_L^2,\sigma_R^2)\) covariance matrix of the noise of the read pulses\textasciigrave{}.

\item {} 
\sphinxAtStartPar
\sphinxstylestrong{wheelBase} : Object attribute containing the distance between the wheels of the robot (\(w=0.5\) m)

\item {} 
\sphinxAtStartPar
\sphinxstylestrong{wheelRadius} : Object attribute containing the radius of the wheels of the robot (\(R=0.1\) m)

\item {} 
\sphinxAtStartPar
\sphinxstylestrong{pulses\_x\_wheelTurn} : Object attribute containing the number of pulses per wheel turn (\(pulseXwheelTurn=1024\) pulses)

\item {} 
\sphinxAtStartPar
\sphinxstylestrong{Polar2D\_max\_range} : Object attribute containing the maximum Polar2D range (\(Polar2D_max_range=50\) m) at which the robot can detect features.

\item {} 
\sphinxAtStartPar
\sphinxstylestrong{Polar2D\_feature\_reading\_frequency} : Object attribute containing the frequency of Polar2D feature readings (50 tics \sphinxhyphen{}sample times\sphinxhyphen{})

\item {} 
\sphinxAtStartPar
\sphinxstylestrong{Rfp} : Object attribute containing the covariance of the simulated Polar2D feature noise (\(R_{fp}=diag(\sigma_{\rho}^2,\sigma_{\phi}^2)\))

\end{itemize}

\sphinxAtStartPar
Check the parent class \sphinxcode{\sphinxupquote{prpy.SimulatedRobot}} to know the rest of the object attributes.

\end{fulllineitems}

\index{fs() (DifferentialDriveSimulatedRobot.DifferentialDriveSimulatedRobot method)@\spxentry{fs()}\spxextra{DifferentialDriveSimulatedRobot.DifferentialDriveSimulatedRobot method}}

\begin{fulllineitems}
\phantomsection\label{\detokenize{robot_simulation:DifferentialDriveSimulatedRobot.DifferentialDriveSimulatedRobot.fs}}\pysiglinewithargsret{\sphinxbfcode{\sphinxupquote{fs}}}{\emph{\DUrole{n}{xsk\_1}}, \emph{\DUrole{n}{usk}}}{}
\sphinxAtStartPar
Motion model used to simulate the robot motion. Computes the current robot state \(x_k\) given the previous robot state \(x_{k-1}\) and the input \(u_k\):
\begin{equation}\label{equation:robot_simulation:eq:fs}
\begin{split}\eta_{s_{k-1}}&=\begin{bmatrix}x_{s_{k-1}} & y_{s_{k-1}} & \theta_{s_{k-1}}\end{bmatrix}^T\\
\nu_{s_{k-1}}&=\begin{bmatrix} u_{s_{k-1}} &  v_{s_{k-1}} & r_{s_{k-1}}\end{bmatrix}^T\\
x_{s_{k-1}}&=\begin{bmatrix}\eta_{s_{k-1}}^T & \nu_{s_{k-1}}^T\end{bmatrix}^T\\
u_{s_k}&=\nu_{d}=\begin{bmatrix} u_d& r_d\end{bmatrix}^T\\
w_{s_k}&=\dot \nu_{s_k}\\
x_{s_k}&=f_s(x_{s_{k-1}},u_{s_k},w_{s_k}) \\
&=\begin{bmatrix}
\eta_{s_{k-1}} \oplus (\nu_{s_{k-1}}\Delta t + \frac{1}{2} w_{s_k}) \\
\nu_{s_{k-1}}+K(\nu_{d}-\nu_{s_{k-1}}) + w_{s_k} \Delta t
\end{bmatrix} \quad;\quad K=diag(k_1,k_2,k_3) \quad k_i>0\\\end{split}
\end{equation}
\sphinxAtStartPar
Where \(\eta_{s_{k-1}}\) is the previous 3 DOF robot pose (x,y,yaw) and \(\nu_{s_{k-1}}\) is the previous robot velocity (velocity in the direction of x and y B\sphinxhyphen{}Frame axis of the robot and the angular velocity).
\(u_{s_k}\) is the input to the motion model contaning the desired robot velocity in the x direction (\(u_d\)) and the desired angular velocity around the z axis (\(r_d\)).
\(w_{s_k}\) is the motion model noise representing an acceleration perturbation in the robot axis. The \(w_{s_k}\) acceleration is the responsible for the slight velocity variation in the simulated robot motion.
\(K\) is a diagonal matrix containing the gains used to drive the simulated velocity towards the desired input velocity.

\sphinxAtStartPar
Finally, the class updates the object attributes \(xsk\), \(xsk\_1\) and  \(usk\) to made them available for plotting purposes.

\sphinxAtStartPar
\sphinxstylestrong{To be completed by the student}.
\begin{quote}\begin{description}
\item[{Parameters}] \leavevmode\begin{itemize}
\item {} 
\sphinxAtStartPar
\sphinxstyleliteralstrong{\sphinxupquote{xsk\_1}} \textendash{} previous robot state \(x_{s_{k-1}}=\begin{bmatrix}\eta_{s_{k-1}}^T & \nu_{s_{k-1}}^T\end{bmatrix}^T\)

\item {} 
\sphinxAtStartPar
\sphinxstyleliteralstrong{\sphinxupquote{usk}} \textendash{} model input \(u_{s_k}=\nu_{d}=\begin{bmatrix} u_d& r_d\end{bmatrix}^T\)

\end{itemize}

\item[{Returns}] \leavevmode
\sphinxAtStartPar
current robot state \(x_{s_k}\)

\end{description}\end{quote}

\end{fulllineitems}

\index{ReadEncoders() (DifferentialDriveSimulatedRobot.DifferentialDriveSimulatedRobot method)@\spxentry{ReadEncoders()}\spxextra{DifferentialDriveSimulatedRobot.DifferentialDriveSimulatedRobot method}}

\begin{fulllineitems}
\phantomsection\label{\detokenize{robot_simulation:DifferentialDriveSimulatedRobot.DifferentialDriveSimulatedRobot.ReadEncoders}}\pysiglinewithargsret{\sphinxbfcode{\sphinxupquote{ReadEncoders}}}{}{}
\sphinxAtStartPar
Simulates the robot measurements of the left and right wheel encoders.

\sphinxAtStartPar
\sphinxstylestrong{To be completed by the student}.
\begin{quote}\begin{description}
\item[{Return zsk,Rsk}] \leavevmode
\sphinxAtStartPar
\(zk=[n_L~n_R]^T\) observation vector containing number of pulses read from the left and right wheel encoders. \(R_{s_k}=diag(\sigma_L^2,\sigma_R^2)\) covariance matrix of the read pulses.

\end{description}\end{quote}

\end{fulllineitems}

\index{ReadCompass() (DifferentialDriveSimulatedRobot.DifferentialDriveSimulatedRobot method)@\spxentry{ReadCompass()}\spxextra{DifferentialDriveSimulatedRobot.DifferentialDriveSimulatedRobot method}}

\begin{fulllineitems}
\phantomsection\label{\detokenize{robot_simulation:DifferentialDriveSimulatedRobot.DifferentialDriveSimulatedRobot.ReadCompass}}\pysiglinewithargsret{\sphinxbfcode{\sphinxupquote{ReadCompass}}}{}{}
\sphinxAtStartPar
Simulates the compass reading of the robot.
\begin{quote}\begin{description}
\item[{Returns}] \leavevmode
\sphinxAtStartPar
yaw and the covariance of its noise \sphinxstyleemphasis{R\_yaw}

\end{description}\end{quote}

\end{fulllineitems}

\index{PlotRobot() (DifferentialDriveSimulatedRobot.DifferentialDriveSimulatedRobot method)@\spxentry{PlotRobot()}\spxextra{DifferentialDriveSimulatedRobot.DifferentialDriveSimulatedRobot method}}

\begin{fulllineitems}
\phantomsection\label{\detokenize{robot_simulation:DifferentialDriveSimulatedRobot.DifferentialDriveSimulatedRobot.PlotRobot}}\pysiglinewithargsret{\sphinxbfcode{\sphinxupquote{PlotRobot}}}{}{}
\sphinxAtStartPar
Updates the plot of the robot at the current pose

\end{fulllineitems}


\end{fulllineitems}



\section{Robot Localization}
\label{\detokenize{Localization_index:robot-localization}}\label{\detokenize{Localization_index::doc}}

\subsection{Robot Localization}
\label{\detokenize{Localization:robot-localization}}\label{\detokenize{Localization::doc}}
\begin{figure}[htbp]
\centering

\noindent\sphinxincludegraphics[scale=0.75]{{Localization}.png}
\end{figure}
\index{Localization (class in Localization)@\spxentry{Localization}\spxextra{class in Localization}}

\begin{fulllineitems}
\phantomsection\label{\detokenize{Localization:Localization.Localization}}\pysiglinewithargsret{\sphinxbfcode{\sphinxupquote{class\DUrole{w}{  }}}\sphinxcode{\sphinxupquote{Localization.}}\sphinxbfcode{\sphinxupquote{Localization}}}{\emph{\DUrole{n}{index}}, \emph{\DUrole{n}{kSteps}}, \emph{\DUrole{n}{robot}}, \emph{\DUrole{n}{x0}}, \emph{\DUrole{o}{*}\DUrole{n}{args}}}{}
\sphinxAtStartPar
Bases: \sphinxcode{\sphinxupquote{object}}

\sphinxAtStartPar
Localization base class. Implements the localization algorithm.
\index{\_\_init\_\_() (Localization.Localization method)@\spxentry{\_\_init\_\_()}\spxextra{Localization.Localization method}}

\begin{fulllineitems}
\phantomsection\label{\detokenize{Localization:Localization.Localization.__init__}}\pysiglinewithargsret{\sphinxbfcode{\sphinxupquote{\_\_init\_\_}}}{\emph{\DUrole{n}{index}}, \emph{\DUrole{n}{kSteps}}, \emph{\DUrole{n}{robot}}, \emph{\DUrole{n}{x0}}, \emph{\DUrole{o}{*}\DUrole{n}{args}}}{}
\sphinxAtStartPar
Constructor of the DRLocalization class.
\begin{quote}\begin{description}
\item[{Parameters}] \leavevmode\begin{itemize}
\item {} 
\sphinxAtStartPar
\sphinxstyleliteralstrong{\sphinxupquote{index}} \textendash{} Logging index structure (\sphinxcode{\sphinxupquote{prpy.Index}})

\item {} 
\sphinxAtStartPar
\sphinxstyleliteralstrong{\sphinxupquote{kSteps}} \textendash{} Number of time steps to simulate

\item {} 
\sphinxAtStartPar
\sphinxstyleliteralstrong{\sphinxupquote{robot}} \textendash{} Simulation robot object (\sphinxcode{\sphinxupquote{prpy.Robot}})

\item {} 
\sphinxAtStartPar
\sphinxstyleliteralstrong{\sphinxupquote{args}} \textendash{} Rest of arguments to be passed to the parent constructor

\item {} 
\sphinxAtStartPar
\sphinxstyleliteralstrong{\sphinxupquote{x0}} \textendash{} Initial Robot pose in the N\sphinxhyphen{}Frame

\end{itemize}

\end{description}\end{quote}

\end{fulllineitems}

\index{GetInput() (Localization.Localization method)@\spxentry{GetInput()}\spxextra{Localization.Localization method}}

\begin{fulllineitems}
\phantomsection\label{\detokenize{Localization:Localization.Localization.GetInput}}\pysiglinewithargsret{\sphinxbfcode{\sphinxupquote{GetInput}}}{}{}
\sphinxAtStartPar
Gets the input from the robot. To be overidden by the child class.
\begin{quote}\begin{description}
\item[{Return uk}] \leavevmode
\sphinxAtStartPar
input variable

\end{description}\end{quote}

\end{fulllineitems}

\index{Localize() (Localization.Localization method)@\spxentry{Localize()}\spxextra{Localization.Localization method}}

\begin{fulllineitems}
\phantomsection\label{\detokenize{Localization:Localization.Localization.Localize}}\pysiglinewithargsret{\sphinxbfcode{\sphinxupquote{Localize}}}{\emph{\DUrole{n}{xk\_1}}, \emph{\DUrole{n}{uk}}}{}
\sphinxAtStartPar
Single Localization iteration invoked from \sphinxcode{\sphinxupquote{prpy.DRLocalization.Localization()}}. Given the previous robot pose, the function reads the inout and computes the current pose.
\begin{quote}\begin{description}
\item[{Parameters}] \leavevmode
\sphinxAtStartPar
\sphinxstyleliteralstrong{\sphinxupquote{xk\_1}} \textendash{} previous robot pose

\item[{Return xk}] \leavevmode
\sphinxAtStartPar
current robot pose

\end{description}\end{quote}

\end{fulllineitems}

\index{LocalizationLoop() (Localization.Localization method)@\spxentry{LocalizationLoop()}\spxextra{Localization.Localization method}}

\begin{fulllineitems}
\phantomsection\label{\detokenize{Localization:Localization.Localization.LocalizationLoop}}\pysiglinewithargsret{\sphinxbfcode{\sphinxupquote{LocalizationLoop}}}{\emph{\DUrole{n}{x0}}, \emph{\DUrole{n}{usk}}}{}
\sphinxAtStartPar
Given an initial robot pose \(x_0\) and the input to the \sphinxcode{\sphinxupquote{prpy.SimulatedRobot}} this method calls iteratively \sphinxcode{\sphinxupquote{prpy.DRLocalization.Localize()}} for k steps, solving the robot localization problem.
\begin{quote}\begin{description}
\item[{Parameters}] \leavevmode
\sphinxAtStartPar
\sphinxstyleliteralstrong{\sphinxupquote{x0}} \textendash{} initial robot pose

\end{description}\end{quote}

\end{fulllineitems}

\index{Log() (Localization.Localization method)@\spxentry{Log()}\spxextra{Localization.Localization method}}

\begin{fulllineitems}
\phantomsection\label{\detokenize{Localization:Localization.Localization.Log}}\pysiglinewithargsret{\sphinxbfcode{\sphinxupquote{Log}}}{\emph{\DUrole{n}{xsk}}, \emph{\DUrole{n}{xk}}}{}
\sphinxAtStartPar
Logs the results for later plotting.
\begin{quote}\begin{description}
\item[{Parameters}] \leavevmode\begin{itemize}
\item {} 
\sphinxAtStartPar
\sphinxstyleliteralstrong{\sphinxupquote{xsk}} \textendash{} ground truth robot pose from the simulation

\item {} 
\sphinxAtStartPar
\sphinxstyleliteralstrong{\sphinxupquote{xk}} \textendash{} estimated robot pose

\end{itemize}

\end{description}\end{quote}

\end{fulllineitems}

\index{PlotXY() (Localization.Localization method)@\spxentry{PlotXY()}\spxextra{Localization.Localization method}}

\begin{fulllineitems}
\phantomsection\label{\detokenize{Localization:Localization.Localization.PlotXY}}\pysiglinewithargsret{\sphinxbfcode{\sphinxupquote{PlotXY}}}{}{}
\sphinxAtStartPar
Plots, in a new figure, the ground truth (orange) and estimated (blue) trajectory of the robot at the end of the Localization Loop.

\end{fulllineitems}

\index{PlotTrajectory() (Localization.Localization method)@\spxentry{PlotTrajectory()}\spxextra{Localization.Localization method}}

\begin{fulllineitems}
\phantomsection\label{\detokenize{Localization:Localization.Localization.PlotTrajectory}}\pysiglinewithargsret{\sphinxbfcode{\sphinxupquote{PlotTrajectory}}}{}{}
\sphinxAtStartPar
Plots the estimated trajectory (blue) of the robot during the localization process.

\end{fulllineitems}


\end{fulllineitems}



\subsection{Dead Reckoning}
\label{\detokenize{Localization_index:dead-reckoning}}

\subsubsection{3 DOF Differential Drive Mobile Robot Example}
\label{\detokenize{DRLocalization:dof-differential-drive-mobile-robot-example}}\label{\detokenize{DRLocalization::doc}}
\begin{figure}[htbp]
\centering

\noindent\sphinxincludegraphics[scale=0.75]{{DR_3DOFDifferentialDrive}.png}
\end{figure}
\index{DR\_3DOFDifferentialDrive (class in DR\_3DOFDifferentialDrive)@\spxentry{DR\_3DOFDifferentialDrive}\spxextra{class in DR\_3DOFDifferentialDrive}}

\begin{fulllineitems}
\phantomsection\label{\detokenize{DRLocalization:DR_3DOFDifferentialDrive.DR_3DOFDifferentialDrive}}\pysiglinewithargsret{\sphinxbfcode{\sphinxupquote{class\DUrole{w}{  }}}\sphinxcode{\sphinxupquote{DR\_3DOFDifferentialDrive.}}\sphinxbfcode{\sphinxupquote{DR\_3DOFDifferentialDrive}}}{\emph{\DUrole{n}{index}}, \emph{\DUrole{n}{kSteps}}, \emph{\DUrole{n}{robot}}, \emph{\DUrole{n}{x0}}, \emph{\DUrole{o}{*}\DUrole{n}{args}}}{}
\sphinxAtStartPar
Bases: {\hyperref[\detokenize{Localization:Localization.Localization}]{\sphinxcrossref{\sphinxcode{\sphinxupquote{Localization.Localization}}}}}

\sphinxAtStartPar
Dead Reckoning Localization for a Differential Drive Mobile Robot.
\index{\_\_init\_\_() (DR\_3DOFDifferentialDrive.DR\_3DOFDifferentialDrive method)@\spxentry{\_\_init\_\_()}\spxextra{DR\_3DOFDifferentialDrive.DR\_3DOFDifferentialDrive method}}

\begin{fulllineitems}
\phantomsection\label{\detokenize{DRLocalization:DR_3DOFDifferentialDrive.DR_3DOFDifferentialDrive.__init__}}\pysiglinewithargsret{\sphinxbfcode{\sphinxupquote{\_\_init\_\_}}}{\emph{\DUrole{n}{index}}, \emph{\DUrole{n}{kSteps}}, \emph{\DUrole{n}{robot}}, \emph{\DUrole{n}{x0}}, \emph{\DUrole{o}{*}\DUrole{n}{args}}}{}
\sphinxAtStartPar
Constructor of the \sphinxcode{\sphinxupquote{prlab.DR\_3DOFDifferentialDrive}} class.
\begin{quote}\begin{description}
\item[{Parameters}] \leavevmode
\sphinxAtStartPar
\sphinxstyleliteralstrong{\sphinxupquote{args}} \textendash{} Rest of arguments to be passed to the parent constructor

\end{description}\end{quote}

\end{fulllineitems}

\index{Localize() (DR\_3DOFDifferentialDrive.DR\_3DOFDifferentialDrive method)@\spxentry{Localize()}\spxextra{DR\_3DOFDifferentialDrive.DR\_3DOFDifferentialDrive method}}

\begin{fulllineitems}
\phantomsection\label{\detokenize{DRLocalization:DR_3DOFDifferentialDrive.DR_3DOFDifferentialDrive.Localize}}\pysiglinewithargsret{\sphinxbfcode{\sphinxupquote{Localize}}}{\emph{\DUrole{n}{xk\_1}}, \emph{\DUrole{n}{uk}}}{}
\sphinxAtStartPar
Motion model for the 3DOF (\(x_k=[x_{k}~y_{k}~\psi_{k}]^T\)) Differential Drive Mobile robot using as input the readings of the wheel encoders (\(u_k=[n_L~n_R]^T\)).
\begin{quote}\begin{description}
\item[{Parameters}] \leavevmode\begin{itemize}
\item {} 
\sphinxAtStartPar
\sphinxstyleliteralstrong{\sphinxupquote{xk\_1}} \textendash{} previous robot pose estimate (\(x_{k-1}=[x_{k-1}~y_{k-1}~\psi_{k-1}]^T\))

\item {} 
\sphinxAtStartPar
\sphinxstyleliteralstrong{\sphinxupquote{uk}} \textendash{} input vector (\(u_k=[u_{k}~v_{k}~w_{k}~r_{k}]^T\))

\end{itemize}

\item[{Return xk}] \leavevmode
\sphinxAtStartPar
current robot pose estimate (\(x_k=[x_{k}~y_{k}~\psi_{k}]^T\))

\end{description}\end{quote}

\end{fulllineitems}

\index{GetInput() (DR\_3DOFDifferentialDrive.DR\_3DOFDifferentialDrive method)@\spxentry{GetInput()}\spxextra{DR\_3DOFDifferentialDrive.DR\_3DOFDifferentialDrive method}}

\begin{fulllineitems}
\phantomsection\label{\detokenize{DRLocalization:DR_3DOFDifferentialDrive.DR_3DOFDifferentialDrive.GetInput}}\pysiglinewithargsret{\sphinxbfcode{\sphinxupquote{GetInput}}}{}{}
\sphinxAtStartPar
Get the input for the motion model. In this case, the input is the readings from both wheel encoders.
\begin{quote}\begin{description}
\item[{Returns}] \leavevmode
\sphinxAtStartPar
uk:  input vector (\(u_k=[n_L~n_R]^T\))

\end{description}\end{quote}

\end{fulllineitems}


\end{fulllineitems}



\section{Particle Filter}
\label{\detokenize{particle_filter:particle-filter}}\label{\detokenize{particle_filter::doc}}
\begin{figure}[htbp]
\centering
\capstart

\noindent\sphinxincludegraphics[scale=0.75]{{ParticleFilter}.png}
\caption{ParticleFilter Class Diagram.}\label{\detokenize{particle_filter:id1}}\end{figure}
\index{ParticleFilter (class in ParticleFilter)@\spxentry{ParticleFilter}\spxextra{class in ParticleFilter}}

\begin{fulllineitems}
\phantomsection\label{\detokenize{particle_filter:ParticleFilter.ParticleFilter}}\pysiglinewithargsret{\sphinxbfcode{\sphinxupquote{class\DUrole{w}{  }}}\sphinxcode{\sphinxupquote{ParticleFilter.}}\sphinxbfcode{\sphinxupquote{ParticleFilter}}}{\emph{\DUrole{n}{index}}, \emph{\DUrole{n}{kSteps}}, \emph{\DUrole{n}{robot}}, \emph{\DUrole{n}{particles}}, \emph{\DUrole{o}{*}\DUrole{n}{args}}}{}
\sphinxAtStartPar
Bases: {\hyperref[\detokenize{Localization:Localization.Localization}]{\sphinxcrossref{\sphinxcode{\sphinxupquote{Localization.Localization}}}}}

\sphinxAtStartPar
Particle Filter Localization.

\sphinxAtStartPar
This class implements basic plotting and logging functionality for the Particle Filter,
as well as the interface for the child classes to implement.

\sphinxAtStartPar
A particle filter is a Monte Carlo algorithm that approximates the posterior distribution of the robot
by a set of weighted particles.  Note that the “weight” (which is a terrible term) is simply the 
probability of the particle being correct. Therefore, each particle is an estimate, and each estimate 
has some probability of being correct.
\index{\_\_init\_\_() (ParticleFilter.ParticleFilter method)@\spxentry{\_\_init\_\_()}\spxextra{ParticleFilter.ParticleFilter method}}

\begin{fulllineitems}
\phantomsection\label{\detokenize{particle_filter:ParticleFilter.ParticleFilter.__init__}}\pysiglinewithargsret{\sphinxbfcode{\sphinxupquote{\_\_init\_\_}}}{\emph{\DUrole{n}{index}}, \emph{\DUrole{n}{kSteps}}, \emph{\DUrole{n}{robot}}, \emph{\DUrole{n}{particles}}, \emph{\DUrole{o}{*}\DUrole{n}{args}}}{}
\sphinxAtStartPar
Constructor of the Particle Filter class.
\begin{quote}\begin{description}
\item[{Parameters}] \leavevmode\begin{itemize}
\item {} 
\sphinxAtStartPar
\sphinxstyleliteralstrong{\sphinxupquote{index}} \textendash{} Logging index structure (\sphinxcode{\sphinxupquote{Index}})

\item {} 
\sphinxAtStartPar
\sphinxstyleliteralstrong{\sphinxupquote{kSteps}} \textendash{} Number of time steps to simulate

\item {} 
\sphinxAtStartPar
\sphinxstyleliteralstrong{\sphinxupquote{robot}} \textendash{} Simulation robot object (\sphinxcode{\sphinxupquote{Robot}})

\item {} 
\sphinxAtStartPar
\sphinxstyleliteralstrong{\sphinxupquote{particles}} \textendash{} initial particles as a list of Pose objects (or at least a list of numpy arrays)

\item {} 
\sphinxAtStartPar
\sphinxstyleliteralstrong{\sphinxupquote{args}} \textendash{} Rest of arguments to be passed to the parent constructor

\end{itemize}

\end{description}\end{quote}

\end{fulllineitems}

\index{MotionModel() (ParticleFilter.ParticleFilter method)@\spxentry{MotionModel()}\spxextra{ParticleFilter.ParticleFilter method}}

\begin{fulllineitems}
\phantomsection\label{\detokenize{particle_filter:ParticleFilter.ParticleFilter.MotionModel}}\pysiglinewithargsret{\sphinxbfcode{\sphinxupquote{MotionModel}}}{\emph{\DUrole{n}{particle}}, \emph{\DUrole{n}{u}}, \emph{\DUrole{n}{noise}}}{}
\sphinxAtStartPar
”
Motion model of the Particle Filter to be overwritten by the child class.
\begin{quote}\begin{description}
\item[{Parameters}] \leavevmode\begin{itemize}
\item {} 
\sphinxAtStartPar
\sphinxstyleliteralstrong{\sphinxupquote{particle}} \textendash{} particle state vector

\item {} 
\sphinxAtStartPar
\sphinxstyleliteralstrong{\sphinxupquote{uk}} \textendash{} input vector

\item {} 
\sphinxAtStartPar
\sphinxstyleliteralstrong{\sphinxupquote{noise}} \textendash{} sample from a noise distribution to be added to the input

\end{itemize}

\item[{Return particle}] \leavevmode
\sphinxAtStartPar
updated particle state vector

\end{description}\end{quote}

\end{fulllineitems}

\index{Weight() (ParticleFilter.ParticleFilter method)@\spxentry{Weight()}\spxextra{ParticleFilter.ParticleFilter method}}

\begin{fulllineitems}
\phantomsection\label{\detokenize{particle_filter:ParticleFilter.ParticleFilter.Weight}}\pysiglinewithargsret{\sphinxbfcode{\sphinxupquote{Weight}}}{\emph{\DUrole{n}{z}}, \emph{\DUrole{n}{R}}}{{ $\rightarrow$ None}}
\sphinxAtStartPar
Weight each particle by the liklihood of the particle being correct.
The probability the particle is correct is given by the probability that it is correct given the measurements (z).
\begin{quote}\begin{description}
\item[{Parameters}] \leavevmode\begin{itemize}
\item {} 
\sphinxAtStartPar
\sphinxstyleliteralstrong{\sphinxupquote{z}} \textendash{} measurement vector

\item {} 
\sphinxAtStartPar
\sphinxstyleliteralstrong{\sphinxupquote{R}} \textendash{} measurement noise covariance

\end{itemize}

\item[{Returns}] \leavevmode
\sphinxAtStartPar
None

\end{description}\end{quote}

\end{fulllineitems}

\index{Resample() (ParticleFilter.ParticleFilter method)@\spxentry{Resample()}\spxextra{ParticleFilter.ParticleFilter method}}

\begin{fulllineitems}
\phantomsection\label{\detokenize{particle_filter:ParticleFilter.ParticleFilter.Resample}}\pysiglinewithargsret{\sphinxbfcode{\sphinxupquote{Resample}}}{}{{ $\rightarrow$ None}}
\sphinxAtStartPar
Resample the particles based on their weights to ensure diversity and prevent particle degeneracy.

\sphinxAtStartPar
This function implements the resampling step of a particle filter algorithm. It uses the weights
assigned to each particle to determine their likelihood of being selected. Particles with higher weights
are more likely to be selected, while those with lower weights have a lower chance.

\sphinxAtStartPar
The resampling process helps to maintain a diverse set of particles that better represents the underlying
probability distribution of the system state.

\sphinxAtStartPar
After resampling, the attributes ‘particles’ and ‘weights’ of the ParticleFilter instance are updated
to reflect the new set of particles and their corresponding weights.
\begin{quote}\begin{description}
\item[{Returns}] \leavevmode
\sphinxAtStartPar
None

\end{description}\end{quote}

\end{fulllineitems}

\index{Prediction() (ParticleFilter.ParticleFilter method)@\spxentry{Prediction()}\spxextra{ParticleFilter.ParticleFilter method}}

\begin{fulllineitems}
\phantomsection\label{\detokenize{particle_filter:ParticleFilter.ParticleFilter.Prediction}}\pysiglinewithargsret{\sphinxbfcode{\sphinxupquote{Prediction}}}{\emph{\DUrole{n}{u}}, \emph{\DUrole{n}{Q}}}{}
\sphinxAtStartPar
Predict the next state of the system based on a given motion model.

\sphinxAtStartPar
This function updates the state of each particle by predicting its next state using a motion model.
\begin{quote}\begin{description}
\item[{Parameters}] \leavevmode\begin{itemize}
\item {} 
\sphinxAtStartPar
\sphinxstyleliteralstrong{\sphinxupquote{u}} \textendash{} input vector

\item {} 
\sphinxAtStartPar
\sphinxstyleliteralstrong{\sphinxupquote{Q}} \textendash{} the covariance matrix associated with the input vector

\end{itemize}

\item[{Returns}] \leavevmode
\sphinxAtStartPar
None

\end{description}\end{quote}

\end{fulllineitems}

\index{Update() (ParticleFilter.ParticleFilter method)@\spxentry{Update()}\spxextra{ParticleFilter.ParticleFilter method}}

\begin{fulllineitems}
\phantomsection\label{\detokenize{particle_filter:ParticleFilter.ParticleFilter.Update}}\pysiglinewithargsret{\sphinxbfcode{\sphinxupquote{Update}}}{\emph{\DUrole{n}{z}}, \emph{\DUrole{n}{R}}}{}
\sphinxAtStartPar
Update the particle weights based on sensor measurements and perform resampling.

\sphinxAtStartPar
This function adjusts the weights of particles based on how well they match the sensor measurements.

\sphinxAtStartPar
The updated weights reflect the likelihood of each particle being the true state of the system given
the sensor measurements.

\sphinxAtStartPar
After updating the weights, the function may perform resampling to ensure that particles with higher
weights are more likely to be selected, maintaining diversity and preventing particle degeneracy.
\begin{quote}\begin{description}
\item[{Parameters}] \leavevmode\begin{itemize}
\item {} 
\sphinxAtStartPar
\sphinxstyleliteralstrong{\sphinxupquote{z}} \textendash{} measurement vector

\item {} 
\sphinxAtStartPar
\sphinxstyleliteralstrong{\sphinxupquote{R}} \textendash{} the covariance matrix associated with the measurement vector

\end{itemize}

\end{description}\end{quote}

\end{fulllineitems}

\index{get\_mean\_particle() (ParticleFilter.ParticleFilter method)@\spxentry{get\_mean\_particle()}\spxextra{ParticleFilter.ParticleFilter method}}

\begin{fulllineitems}
\phantomsection\label{\detokenize{particle_filter:ParticleFilter.ParticleFilter.get_mean_particle}}\pysiglinewithargsret{\sphinxbfcode{\sphinxupquote{get\_mean\_particle}}}{}{}
\sphinxAtStartPar
Calculate the mean particle based on the current set of particles and their weights.
:return: mean particle

\end{fulllineitems}

\index{get\_best\_particle() (ParticleFilter.ParticleFilter method)@\spxentry{get\_best\_particle()}\spxextra{ParticleFilter.ParticleFilter method}}

\begin{fulllineitems}
\phantomsection\label{\detokenize{particle_filter:ParticleFilter.ParticleFilter.get_best_particle}}\pysiglinewithargsret{\sphinxbfcode{\sphinxupquote{get\_best\_particle}}}{}{}
\sphinxAtStartPar
Calculate the best particle based on the current set of particles and their weights.
:return: best particle

\end{fulllineitems}

\index{init\_plotting() (ParticleFilter.ParticleFilter method)@\spxentry{init\_plotting()}\spxextra{ParticleFilter.ParticleFilter method}}

\begin{fulllineitems}
\phantomsection\label{\detokenize{particle_filter:ParticleFilter.ParticleFilter.init_plotting}}\pysiglinewithargsret{\sphinxbfcode{\sphinxupquote{init\_plotting}}}{}{}
\sphinxAtStartPar
Init the plotting of the particles and the mean particle.

\end{fulllineitems}

\index{PlotParticles() (ParticleFilter.ParticleFilter method)@\spxentry{PlotParticles()}\spxextra{ParticleFilter.ParticleFilter method}}

\begin{fulllineitems}
\phantomsection\label{\detokenize{particle_filter:ParticleFilter.ParticleFilter.PlotParticles}}\pysiglinewithargsret{\sphinxbfcode{\sphinxupquote{PlotParticles}}}{}{}
\sphinxAtStartPar
Plots all the particles and the mean particle.
Particles are plotted as green dots, and the mean particle is plotted as a blue dot.
Particle orientation is plotted as a green line, and the mean particle orientation is plotted as a blue line.
Particle size is proportional to the particle weight.
Note that the size is scaled for visualization purposes, and does not reflect the actual weight.

\end{fulllineitems}


\end{fulllineitems}



\section{MonteCarlo Localization}
\label{\detokenize{particle_filter:montecarlo-localization}}
\begin{figure}[htbp]
\centering
\capstart

\noindent\sphinxincludegraphics[scale=0.75]{{MCLocalization}.png}
\caption{MCLocalization Class Diagram.}\label{\detokenize{particle_filter:id2}}\end{figure}
\index{MCLocalization (class in MCLocalization)@\spxentry{MCLocalization}\spxextra{class in MCLocalization}}

\begin{fulllineitems}
\phantomsection\label{\detokenize{particle_filter:MCLocalization.MCLocalization}}\pysiglinewithargsret{\sphinxbfcode{\sphinxupquote{class\DUrole{w}{  }}}\sphinxcode{\sphinxupquote{MCLocalization.}}\sphinxbfcode{\sphinxupquote{MCLocalization}}}{\emph{\DUrole{n}{index}}, \emph{\DUrole{n}{kSteps}}, \emph{\DUrole{n}{robot}}, \emph{\DUrole{n}{particles}}, \emph{\DUrole{o}{*}\DUrole{n}{args}}}{}
\sphinxAtStartPar
Bases: {\hyperref[\detokenize{particle_filter:ParticleFilter.ParticleFilter}]{\sphinxcrossref{\sphinxcode{\sphinxupquote{ParticleFilter.ParticleFilter}}}}}

\sphinxAtStartPar
Monte Carlo Localization class.

\sphinxAtStartPar
This class is used as “Dead Reckoning” localization using a Particle Filter.
It implements the Prediction method from \sphinxcode{\sphinxupquote{ParticleFilter}} and the 
Localize and LocalizationLoop methods from \sphinxcode{\sphinxupquote{Localization}}.
\index{\_\_init\_\_() (MCLocalization.MCLocalization method)@\spxentry{\_\_init\_\_()}\spxextra{MCLocalization.MCLocalization method}}

\begin{fulllineitems}
\phantomsection\label{\detokenize{particle_filter:MCLocalization.MCLocalization.__init__}}\pysiglinewithargsret{\sphinxbfcode{\sphinxupquote{\_\_init\_\_}}}{\emph{\DUrole{n}{index}}, \emph{\DUrole{n}{kSteps}}, \emph{\DUrole{n}{robot}}, \emph{\DUrole{n}{particles}}, \emph{\DUrole{o}{*}\DUrole{n}{args}}}{}
\sphinxAtStartPar
Constructor.
:param index: Named tuple used to map the state vector, the simulation vector and the observation vector (\sphinxcode{\sphinxupquote{prpy.IndexStruct}})
:param kSteps: simulation time steps
:param robot: Simulated Robot object
:param particles: initial particles as a list of Pose objects (or at least a list of numpy arrays)
:param args: arguments to be passed to the parent constructor

\end{fulllineitems}

\index{Prediction() (MCLocalization.MCLocalization method)@\spxentry{Prediction()}\spxextra{MCLocalization.MCLocalization method}}

\begin{fulllineitems}
\phantomsection\label{\detokenize{particle_filter:MCLocalization.MCLocalization.Prediction}}\pysiglinewithargsret{\sphinxbfcode{\sphinxupquote{Prediction}}}{\emph{\DUrole{n}{u}}, \emph{\DUrole{n}{Q}}}{}
\sphinxAtStartPar
Prediction overriden from \sphinxcode{\sphinxupquote{ParticleFilter}}.
Note: Use the MotionModel method from \sphinxcode{\sphinxupquote{ParticleFilter}} to update the particles to keep it generic. Then,
child classes can overwrite the MotionModel method to implement their own motion model.

\end{fulllineitems}

\index{Localize() (MCLocalization.MCLocalization method)@\spxentry{Localize()}\spxextra{MCLocalization.MCLocalization method}}

\begin{fulllineitems}
\phantomsection\label{\detokenize{particle_filter:MCLocalization.MCLocalization.Localize}}\pysiglinewithargsret{\sphinxbfcode{\sphinxupquote{Localize}}}{}{}
\sphinxAtStartPar
Single Localization iteration. Given the previous robot pose, the function reads the inout and computes the current pose.
\begin{quote}\begin{description}
\item[{Returns}] \leavevmode
\sphinxAtStartPar
\sphinxstylestrong{xk} current robot pose (we can assume the mean of the particles or the most likely particle)

\end{description}\end{quote}

\end{fulllineitems}

\index{LocalizationLoop() (MCLocalization.MCLocalization method)@\spxentry{LocalizationLoop()}\spxextra{MCLocalization.MCLocalization method}}

\begin{fulllineitems}
\phantomsection\label{\detokenize{particle_filter:MCLocalization.MCLocalization.LocalizationLoop}}\pysiglinewithargsret{\sphinxbfcode{\sphinxupquote{LocalizationLoop}}}{\emph{\DUrole{n}{x0}}, \emph{\DUrole{n}{usk}}}{}
\sphinxAtStartPar
Given an initial robot pose \(x_0\) and the input to the \sphinxcode{\sphinxupquote{SimulatedRobot}} this method calls iteratively \sphinxcode{\sphinxupquote{DRLocalization.Localize()}} for k steps, solving the robot localization problem.
\begin{quote}\begin{description}
\item[{Parameters}] \leavevmode\begin{itemize}
\item {} 
\sphinxAtStartPar
\sphinxstyleliteralstrong{\sphinxupquote{x0}} \textendash{} initial robot pose

\item {} 
\sphinxAtStartPar
\sphinxstyleliteralstrong{\sphinxupquote{usk}} \textendash{} input vector for the simulation

\end{itemize}

\end{description}\end{quote}

\end{fulllineitems}


\end{fulllineitems}



\section{Particle Filter Map Based Localization}
\label{\detokenize{particle_filter:particle-filter-map-based-localization}}
\begin{figure}[htbp]
\centering
\capstart

\noindent\sphinxincludegraphics[scale=0.75]{{PFMBL}.png}
\caption{PFMBL Class Diagram.}\label{\detokenize{particle_filter:id3}}\end{figure}
\index{PFMBL (class in PFMBLocalization)@\spxentry{PFMBL}\spxextra{class in PFMBLocalization}}

\begin{fulllineitems}
\phantomsection\label{\detokenize{particle_filter:PFMBLocalization.PFMBL}}\pysiglinewithargsret{\sphinxbfcode{\sphinxupquote{class\DUrole{w}{  }}}\sphinxcode{\sphinxupquote{PFMBLocalization.}}\sphinxbfcode{\sphinxupquote{PFMBL}}}{\emph{\DUrole{n}{zf\_dim}}, \emph{\DUrole{n}{M}}, \emph{\DUrole{o}{*}\DUrole{n}{args}}}{}
\sphinxAtStartPar
Bases: {\hyperref[\detokenize{particle_filter:MCLocalization.MCLocalization}]{\sphinxcrossref{\sphinxcode{\sphinxupquote{MCLocalization.MCLocalization}}}}}

\sphinxAtStartPar
Particle Filter Map Based Localization class.

\sphinxAtStartPar
This class defines a Map Based Localization using a Particle Filter. It inherits from \sphinxcode{\sphinxupquote{MCLocalization}}, so the Prediction step is already implemented.
It needs to implement the Update function, and consecuently the Weight and Resample functions.
\index{\_\_init\_\_() (PFMBLocalization.PFMBL method)@\spxentry{\_\_init\_\_()}\spxextra{PFMBLocalization.PFMBL method}}

\begin{fulllineitems}
\phantomsection\label{\detokenize{particle_filter:PFMBLocalization.PFMBL.__init__}}\pysiglinewithargsret{\sphinxbfcode{\sphinxupquote{\_\_init\_\_}}}{\emph{\DUrole{n}{zf\_dim}}, \emph{\DUrole{n}{M}}, \emph{\DUrole{o}{*}\DUrole{n}{args}}}{{ $\rightarrow$ None}}
\sphinxAtStartPar
Constructor.
:param index: Named tuple used to map the state vector, the simulation vector and the observation vector (\sphinxcode{\sphinxupquote{prpy.IndexStruct}})
:param kSteps: simulation time steps
:param robot: Simulated Robot object
:param particles: initial particles as a list of Pose objects (or at least a list of numpy arrays)
:param args: arguments to be passed to the parent constructor

\end{fulllineitems}

\index{Weight() (PFMBLocalization.PFMBL method)@\spxentry{Weight()}\spxextra{PFMBLocalization.PFMBL method}}

\begin{fulllineitems}
\phantomsection\label{\detokenize{particle_filter:PFMBLocalization.PFMBL.Weight}}\pysiglinewithargsret{\sphinxbfcode{\sphinxupquote{Weight}}}{\emph{\DUrole{n}{z}}, \emph{\DUrole{n}{R}}}{}
\sphinxAtStartPar
Weight each particle by the liklihood of the particle being correct.
The probability the particle is correct is given by the probability that it is correct given the measurements (z).
\begin{quote}\begin{description}
\item[{Parameters}] \leavevmode\begin{itemize}
\item {} 
\sphinxAtStartPar
\sphinxstyleliteralstrong{\sphinxupquote{z}} \textendash{} measurement vector

\item {} 
\sphinxAtStartPar
\sphinxstyleliteralstrong{\sphinxupquote{R}} \textendash{} measurement noise covariance

\end{itemize}

\item[{Returns}] \leavevmode
\sphinxAtStartPar
None

\end{description}\end{quote}

\end{fulllineitems}

\index{Resample() (PFMBLocalization.PFMBL method)@\spxentry{Resample()}\spxextra{PFMBLocalization.PFMBL method}}

\begin{fulllineitems}
\phantomsection\label{\detokenize{particle_filter:PFMBLocalization.PFMBL.Resample}}\pysiglinewithargsret{\sphinxbfcode{\sphinxupquote{Resample}}}{}{}
\sphinxAtStartPar
Resample the particles based on their weights to ensure diversity and prevent particle degeneracy.

\sphinxAtStartPar
This function implements the resampling step of a particle filter algorithm. It uses the weights
assigned to each particle to determine their likelihood of being selected. Particles with higher weights
are more likely to be selected, while those with lower weights have a lower chance.

\sphinxAtStartPar
The resampling process helps to maintain a diverse set of particles that better represents the underlying
probability distribution of the system state.

\sphinxAtStartPar
After resampling, the attributes ‘particles’ and ‘weights’ of the ParticleFilter instance are updated
to reflect the new set of particles and their corresponding weights.
\begin{quote}\begin{description}
\item[{Returns}] \leavevmode
\sphinxAtStartPar
None

\end{description}\end{quote}

\end{fulllineitems}

\index{Update() (PFMBLocalization.PFMBL method)@\spxentry{Update()}\spxextra{PFMBLocalization.PFMBL method}}

\begin{fulllineitems}
\phantomsection\label{\detokenize{particle_filter:PFMBLocalization.PFMBL.Update}}\pysiglinewithargsret{\sphinxbfcode{\sphinxupquote{Update}}}{\emph{\DUrole{n}{z}}, \emph{\DUrole{n}{R}}}{}
\sphinxAtStartPar
Update the particle weights based on sensor measurements and perform resampling.

\sphinxAtStartPar
This function adjusts the weights of particles based on how well they match the sensor measurements.

\sphinxAtStartPar
The updated weights reflect the likelihood of each particle being the true state of the system given
the sensor measurements.

\sphinxAtStartPar
After updating the weights, the function may perform resampling to ensure that particles with higher
weights are more likely to be selected, maintaining diversity and preventing particle degeneracy.
\begin{quote}\begin{description}
\item[{Parameters}] \leavevmode\begin{itemize}
\item {} 
\sphinxAtStartPar
\sphinxstyleliteralstrong{\sphinxupquote{z}} \textendash{} measurement vector

\item {} 
\sphinxAtStartPar
\sphinxstyleliteralstrong{\sphinxupquote{R}} \textendash{} the covariance matrix associated with the measurement vector

\end{itemize}

\end{description}\end{quote}

\end{fulllineitems}

\index{Localize() (PFMBLocalization.PFMBL method)@\spxentry{Localize()}\spxextra{PFMBLocalization.PFMBL method}}

\begin{fulllineitems}
\phantomsection\label{\detokenize{particle_filter:PFMBLocalization.PFMBL.Localize}}\pysiglinewithargsret{\sphinxbfcode{\sphinxupquote{Localize}}}{}{}
\sphinxAtStartPar
Single Localization iteration. Given the previous robot pose, the function reads the inout and computes the current pose.
\begin{quote}\begin{description}
\item[{Returns}] \leavevmode
\sphinxAtStartPar
\sphinxstylestrong{xk} current robot pose (we can assume the mean of the particles or the most likely particle)

\end{description}\end{quote}

\end{fulllineitems}


\end{fulllineitems}



\section{PF 3DOF Dead Reckoning}
\label{\detokenize{particle_filter:pf-3dof-dead-reckoning}}
\begin{figure}[htbp]
\centering
\capstart

\noindent\sphinxincludegraphics[scale=0.75]{{PF_3DOF_DR}.png}
\caption{PF\_3DOF\_DR Class Diagram.}\label{\detokenize{particle_filter:id4}}\end{figure}
\index{PF\_3DOF\_DR (class in PF\_3DOF\_DR)@\spxentry{PF\_3DOF\_DR}\spxextra{class in PF\_3DOF\_DR}}

\begin{fulllineitems}
\phantomsection\label{\detokenize{particle_filter:PF_3DOF_DR.PF_3DOF_DR}}\pysiglinewithargsret{\sphinxbfcode{\sphinxupquote{class\DUrole{w}{  }}}\sphinxcode{\sphinxupquote{PF\_3DOF\_DR.}}\sphinxbfcode{\sphinxupquote{PF\_3DOF\_DR}}}{\emph{\DUrole{o}{*}\DUrole{n}{args}}}{}
\sphinxAtStartPar
Bases: {\hyperref[\detokenize{particle_filter:MCLocalization.MCLocalization}]{\sphinxcrossref{\sphinxcode{\sphinxupquote{MCLocalization.MCLocalization}}}}}
\index{\_\_init\_\_() (PF\_3DOF\_DR.PF\_3DOF\_DR method)@\spxentry{\_\_init\_\_()}\spxextra{PF\_3DOF\_DR.PF\_3DOF\_DR method}}

\begin{fulllineitems}
\phantomsection\label{\detokenize{particle_filter:PF_3DOF_DR.PF_3DOF_DR.__init__}}\pysiglinewithargsret{\sphinxbfcode{\sphinxupquote{\_\_init\_\_}}}{\emph{\DUrole{o}{*}\DUrole{n}{args}}}{}
\sphinxAtStartPar
Constructor.
:param index: Named tuple used to map the state vector, the simulation vector and the observation vector (\sphinxcode{\sphinxupquote{prpy.IndexStruct}})
:param kSteps: simulation time steps
:param robot: Simulated Robot object
:param particles: initial particles as a list of Pose objects (or at least a list of numpy arrays)
:param args: arguments to be passed to the parent constructor

\end{fulllineitems}

\index{GetInput() (PF\_3DOF\_DR.PF\_3DOF\_DR method)@\spxentry{GetInput()}\spxextra{PF\_3DOF\_DR.PF\_3DOF\_DR method}}

\begin{fulllineitems}
\phantomsection\label{\detokenize{particle_filter:PF_3DOF_DR.PF_3DOF_DR.GetInput}}\pysiglinewithargsret{\sphinxbfcode{\sphinxupquote{GetInput}}}{}{}
\sphinxAtStartPar
Get the input for the motion model.
\begin{quote}\begin{description}
\item[{Returns}] \leavevmode
\sphinxAtStartPar
\begin{itemize}
\item {} 
\sphinxAtStartPar
\sphinxstylestrong{uk, Qk}. uk: input vector (\(u_k=[n_L~n_R]^T\)), Qk: covariance of the input noise

\end{itemize}


\end{description}\end{quote}

\end{fulllineitems}

\index{MotionModel() (PF\_3DOF\_DR.PF\_3DOF\_DR method)@\spxentry{MotionModel()}\spxextra{PF\_3DOF\_DR.PF\_3DOF\_DR method}}

\begin{fulllineitems}
\phantomsection\label{\detokenize{particle_filter:PF_3DOF_DR.PF_3DOF_DR.MotionModel}}\pysiglinewithargsret{\sphinxbfcode{\sphinxupquote{MotionModel}}}{\emph{\DUrole{n}{particle}}, \emph{\DUrole{n}{u}}, \emph{\DUrole{n}{noise}}}{}
\sphinxAtStartPar
”
Motion model of the Particle Filter to be overwritten by the child class.
\begin{quote}\begin{description}
\item[{Parameters}] \leavevmode\begin{itemize}
\item {} 
\sphinxAtStartPar
\sphinxstyleliteralstrong{\sphinxupquote{particle}} \textendash{} particle state vector

\item {} 
\sphinxAtStartPar
\sphinxstyleliteralstrong{\sphinxupquote{uk}} \textendash{} input vector

\item {} 
\sphinxAtStartPar
\sphinxstyleliteralstrong{\sphinxupquote{noise}} \textendash{} sample from a noise distribution to be added to the input

\end{itemize}

\item[{Return particle}] \leavevmode
\sphinxAtStartPar
updated particle state vector

\end{description}\end{quote}

\end{fulllineitems}


\end{fulllineitems}



\section{PF 3DOF Map Based Localization}
\label{\detokenize{particle_filter:pf-3dof-map-based-localization}}
\begin{figure}[htbp]
\centering
\capstart

\noindent\sphinxincludegraphics[scale=0.75]{{PF_3DOF_MBL}.png}
\caption{PF\_3DOF\_MBL Class Diagram.}\label{\detokenize{particle_filter:id5}}\end{figure}
\index{PF\_3DOF\_MBL (class in PF\_3DOF\_MBL)@\spxentry{PF\_3DOF\_MBL}\spxextra{class in PF\_3DOF\_MBL}}

\begin{fulllineitems}
\phantomsection\label{\detokenize{particle_filter:PF_3DOF_MBL.PF_3DOF_MBL}}\pysiglinewithargsret{\sphinxbfcode{\sphinxupquote{class\DUrole{w}{  }}}\sphinxcode{\sphinxupquote{PF\_3DOF\_MBL.}}\sphinxbfcode{\sphinxupquote{PF\_3DOF\_MBL}}}{\emph{\DUrole{o}{*}\DUrole{n}{args}}}{}
\sphinxAtStartPar
Bases: {\hyperref[\detokenize{particle_filter:PFMBLocalization.PFMBL}]{\sphinxcrossref{\sphinxcode{\sphinxupquote{PFMBLocalization.PFMBL}}}}}
\index{\_\_init\_\_() (PF\_3DOF\_MBL.PF\_3DOF\_MBL method)@\spxentry{\_\_init\_\_()}\spxextra{PF\_3DOF\_MBL.PF\_3DOF\_MBL method}}

\begin{fulllineitems}
\phantomsection\label{\detokenize{particle_filter:PF_3DOF_MBL.PF_3DOF_MBL.__init__}}\pysiglinewithargsret{\sphinxbfcode{\sphinxupquote{\_\_init\_\_}}}{\emph{\DUrole{o}{*}\DUrole{n}{args}}}{}
\sphinxAtStartPar
Constructor.
:param index: Named tuple used to map the state vector, the simulation vector and the observation vector (\sphinxcode{\sphinxupquote{prpy.IndexStruct}})
:param kSteps: simulation time steps
:param robot: Simulated Robot object
:param particles: initial particles as a list of Pose objects (or at least a list of numpy arrays)
:param args: arguments to be passed to the parent constructor

\end{fulllineitems}

\index{GetInput() (PF\_3DOF\_MBL.PF\_3DOF\_MBL method)@\spxentry{GetInput()}\spxextra{PF\_3DOF\_MBL.PF\_3DOF\_MBL method}}

\begin{fulllineitems}
\phantomsection\label{\detokenize{particle_filter:PF_3DOF_MBL.PF_3DOF_MBL.GetInput}}\pysiglinewithargsret{\sphinxbfcode{\sphinxupquote{GetInput}}}{}{}
\sphinxAtStartPar
Get the input for the motion model.
\begin{quote}\begin{description}
\item[{Returns}] \leavevmode
\sphinxAtStartPar
\begin{itemize}
\item {} 
\sphinxAtStartPar
\sphinxstylestrong{uk, Qk}. uk: input vector (\(u_k=[n_L~n_R]^T\)), Qk: covariance of the input noise

\end{itemize}


\end{description}\end{quote}

\end{fulllineitems}

\index{GetMeasurements() (PF\_3DOF\_MBL.PF\_3DOF\_MBL method)@\spxentry{GetMeasurements()}\spxextra{PF\_3DOF\_MBL.PF\_3DOF\_MBL method}}

\begin{fulllineitems}
\phantomsection\label{\detokenize{particle_filter:PF_3DOF_MBL.PF_3DOF_MBL.GetMeasurements}}\pysiglinewithargsret{\sphinxbfcode{\sphinxupquote{GetMeasurements}}}{}{}
\sphinxAtStartPar
Read the measurements from the robot. Returns a vector of range distances to the map features.
Only those features that are within the \sphinxcode{\sphinxupquote{SimulatedRobot.SimulatedRobot.Distance\_max\_range}} of the sensor are returned.
The measurements arribe at a frequency defined in the \sphinxcode{\sphinxupquote{SimulatedRobot.SimulatedRobot.Distance\_feature\_reading\_frequency}} attribute.
\begin{quote}\begin{description}
\item[{Returns}] \leavevmode
\sphinxAtStartPar
vector of distances to the map features, covariance of the measurement noise

\end{description}\end{quote}

\end{fulllineitems}

\index{MotionModel() (PF\_3DOF\_MBL.PF\_3DOF\_MBL method)@\spxentry{MotionModel()}\spxextra{PF\_3DOF\_MBL.PF\_3DOF\_MBL method}}

\begin{fulllineitems}
\phantomsection\label{\detokenize{particle_filter:PF_3DOF_MBL.PF_3DOF_MBL.MotionModel}}\pysiglinewithargsret{\sphinxbfcode{\sphinxupquote{MotionModel}}}{\emph{\DUrole{n}{particle}}, \emph{\DUrole{n}{u}}, \emph{\DUrole{n}{noise}}}{}
\sphinxAtStartPar
”
Motion model of the Particle Filter to be overwritten by the child class.
\begin{quote}\begin{description}
\item[{Parameters}] \leavevmode\begin{itemize}
\item {} 
\sphinxAtStartPar
\sphinxstyleliteralstrong{\sphinxupquote{particle}} \textendash{} particle state vector

\item {} 
\sphinxAtStartPar
\sphinxstyleliteralstrong{\sphinxupquote{uk}} \textendash{} input vector

\item {} 
\sphinxAtStartPar
\sphinxstyleliteralstrong{\sphinxupquote{noise}} \textendash{} sample from a noise distribution to be added to the input

\end{itemize}

\item[{Return particle}] \leavevmode
\sphinxAtStartPar
updated particle state vector

\end{description}\end{quote}

\end{fulllineitems}


\end{fulllineitems}



\begin{savenotes}\sphinxatlongtablestart\begin{longtable}[c]{\X{1}{2}\X{1}{2}}
\hline

\endfirsthead

\multicolumn{2}{c}%
{\makebox[0pt]{\sphinxtablecontinued{\tablename\ \thetable{} \textendash{} continued from previous page}}}\\
\hline

\endhead

\hline
\multicolumn{2}{r}{\makebox[0pt][r]{\sphinxtablecontinued{continues on next page}}}\\
\endfoot

\endlastfoot

\end{longtable}\sphinxatlongtableend\end{savenotes}


\chapter{Indices and tables}
\label{\detokenize{index:indices-and-tables}}\begin{itemize}
\item {} 
\sphinxAtStartPar
\DUrole{xref,std,std-ref}{genindex}

\item {} 
\sphinxAtStartPar
\DUrole{xref,std,std-ref}{modindex}

\item {} 
\sphinxAtStartPar
\DUrole{xref,std,std-ref}{search}

\end{itemize}



\renewcommand{\indexname}{Index}
\printindex
\end{document}